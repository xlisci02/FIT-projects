\documentclass[11pt,a4paper,titlepage]{article}
\usepackage[left=2cm,text={17cm,24cm},top=3cm]{geometry}
\usepackage[T1]{fontenc}
\usepackage[czech]{babel}
\usepackage[utf8]{inputenc}
\usepackage[ddmmyyyy]{datetime}


\newcommand{\myuv}[1]{\quotedblbase #1\textquotedblleft} 
\renewcommand{\dateseparator}{.}
\bibliographystyle{czplain}

\begin{document}

\begin{titlepage}
\begin{center}
{\Huge\textsc{Vysoké učenie technické v~Brne}}\\
\medskip
{\huge\textsc{Fakulta informačných technológií}}\\
\vspace{\stretch{0.382}}
{\LARGE Typografia a~publikovanie\,--\,4.\,projekt}\\
\medskip
{\Huge Bibliografické citácie}\\
\vspace{\stretch{0.618}}
\end{center}

{\Large\today \hfill Matúš Liščinský}
\end{titlepage}

\begin{center}
\textbf{\textsc{\Huge{Typografia}}}
\end{center}

\par 
\textbf{Typografia} je umelecko-technický obor, ktorý sa zaoberá tlačovým písmom \cite{Wiki:Typografie}.

V~dobe pred vynálezom kníhtlače vznikali všetky písomnosti ručne a~väčšinou na nich pracovali mnísi v~skriptóriách \cite{History_of_Typography}. 
O~vynález kníhtlače sa zaslúžil Nemec Johann Gutenberg v~roku 1455 a~prvou knihou vytlačenou na tejto kníhtlači bola tzv. Gutenbergova Biblia \cite{Cin_Magazine}.

S~koncom 20. storočia dochádza k~hromadnému zavádzaniu výpočetnej techniky. Oblasť spracováva\-nia textov nebola výnimkou. Počítač s~vhodným programovým vybavením môže nahradiť v~minulosti
používané technológie a~výrazne tak urýchliť a~uľahčiť proces prípravy a~tlače dokumentov \cite{Uhrik:Bakalar}.
Pri sadzbe textu na počítači sa v~minulosti vedome spáchalo mnoho \myuv{zločinov}. Dôvodom bola obmedzenosť rozsahu
znakových sád, ktorá vychádzala z~možnosti kódovania \cite{Jak_psat}. Znakové sady v~typografických systémoch majú jediný cieľ: kvalitný obraz písma v~tlačenom dokumente \cite{Cerny:Znakove_sady}.

Uplatnenie typografie nájdeme aj mimo dokumentov či kníh. Webové stránky sú tiež príkladom používania typografie. 
V~začiatkoch tvorby webových stránok sa vývojári skôr zameriavali na kód ako na design. Dnes je to však inak. Podobnými problémami z~oblasti typografie sa zaoberal aj časopis Dwell \cite{Dwell}.

Jedným z~obľubených nástrojov na sadzbu textu je \LaTeX{}. Ide vlastne o~systém makier vystavený nad \TeX{}om \cite{Martinek_Latex}. Hlavnou prácou  pri realizácii publikácie v~systéme \LaTeX{} je zápis zdrojového textu spoločne s~príkazmi ovplyvňujúcimi spôsob sadzby \cite{Rybicka:Latex}. V súvislosti s~\LaTeX{}om sa taktiež konajú konferencie zaoberajúce sa práve systémom \LaTeX{}. Obsah konferencií sa zdokumentuje a~vydávajú sa zborníky \cite{Konferencia}. \LaTeX{} je len jedným z~takýchto nástrojov, ďalší rozšírený a~používaný je \textit{Adobe InDesign} umožňujúci aj kontrolu typografických zásad \cite{Kubova_Adobe}.

\newpage
\renewcommand{\refname}{Referencie}
\bibliography{Literatura}


\end{document}

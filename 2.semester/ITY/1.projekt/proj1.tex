\documentclass[11pt,a4paper,twocolumn]{article}
\usepackage[czech]{babel}
\usepackage[utf8]{inputenc}
\usepackage[IL2]{fontenc}
\usepackage{times}
\usepackage[text={17cm,24cm},left=2cm,top=2.5cm]{geometry}
\newcommand{\myuv}[1]{\quotedblbase #1\textquotedblleft}

\begin{document}
\title {Typografie a~publikování\\1.\,projekt }
\date{}
\author {Matúš Liščinský \\ xlisci02@stud.fit.vutbr.cz}
\maketitle

\section {Hladká sazba}
Hladká sazba je sazba z~jednoho stupně, druhu a~řezu pí­sma sázená na stanovenou šířku odstavce. 
Skládá se z~odstavců, které obvykle začínají­ zarážkou, ale mo\-hou být sázeny i~bez zarážky -- rozhodují­cí­ je celková grafická úprava. 
Odstavce jsou ukončeny východovou řádkou. Věty nesmějí začínat číslicí.

Barevné zvýraznění­, podtrhávání­ slov či různé ve\-li\-kos\-ti písma vybraných slov se zde také nepoužívá. Hladká sazba je určena především pro delší­ texty, jako~je napří­klad beletrie. Porušení­ konzistence sazby působí v~textu rušivě a~unavuje čtenářův zrak.

\section {Smíšená sazba}
Smíšená sazba má o~něco volnější­ pravidla než hladká sazba. Nejčastěji se klasická hladká sazba doplňuje o~další řezy pí­sma pro zvýraznění­ důležitých pojmů. Existuje \myuv{pravidlo}:

\begin{quotation}
Čí­m ví­ce \textbf{druhů}, \textbf{\emph{řezů}},  {\scriptsize velikostí}, barev pí\-­sma a~jiných efektů použijeme, tí­m \emph{profe\-sionálněji} bude dokument vypadat. Čtenář tím bude vždy {\Huge nadšen!} 
\end{quotation}

\textsc{Tí­mto{\hfill}pravidlem{\hfill}se{\hfill}\underline{nikdy}{\hfill}nesmí­te{\hfill}ří­dit.\break}
Příliš časté zvýrazňování textových elementů  a~změny velikosti {\tiny pí­sma} jsou {\LARGE známkou} \textbf{{\huge amatéris\-mu}} autora a~působí­ \textbf{\emph{velmi}} \emph{rušivě}. Dobře navrže\-ný dokument nemá obsahovat ví­ce než 4 řezy či druhy pí­sma. \texttt{Dobře navržený dokument je decentní­, ne chaotický.}

Důležitým znakem správně vysázeného dokumentu je konzistentní použí­vání­ různých druhů zvýraznění­. To napří­klad může znamenat, že \textbf{tučný řez} pí­sma bude vyhrazen pouze pro klíčová slova, \emph{skloněný řez} pouze pro doposud neznámé pojmy a~nebude se to míchat. Skloněný řez nepůsobí­ tak rušivě a~použí­vá se častěji.
V~\LaTeX{}u jej sází­me raději pří­kazem \verb|\emph{text}| než \verb|\textit{text}|. 

Smíšená sazba se nejčastěji používá pro sazbu vě\-dec\-kých článků a~technických zpráv. 
U~delší­ch do\-ku\-men\-tů vědeckého či technického charakteru je zvykem upozornit čtenáře na význam různých typů zvýrazně\-ní­ v~úvodní­ kapitole.

\section{České odlišnosti}
Česká sazba se oproti okolnímu světu v~některých as\-pektech mírně liší. Jednou z~odlišností je sazba uvo\-zo\-vek. Uvozovky se v~češtině používají převážně pro zobrazení přímé řeči. V~menší míře se používají také pro zvýraznění přezdívek a~ironie. V~češtině se použí\-vá tento \myuv{\textbf {typ uvozovek}} namísto anglických ``uvo\-zovek''. Lze je sázet připravenými příkazy nebo při použití UTF-8 kódování i~přímo.

Ve smíšené sazbě se řez uvozovek řídí řezem prv\-ní\-ho uvozovaného slova. Pokud je uvozována celá věta, sází se koncová tečka před uvozovku, pokud se uvozuje slovo nebo část věty, sází se tečka za uvozovku.

Druhou odlišností je pravidlo pro sázení­ konců řádků. V~české sazbě by řádek neměl končit osamoce\-nou jednopí­smennou předložkou nebo spojkou. Spoj\-kou \myuv{a} končit může při sazbě do 25 liter. 
Abychom \LaTeX{}u zabránili v~sázení­ osamocených předložek, vkládáme mezi předložku a~slovo \textbf{nezlomitelnou mezeru} znakem \verb|~| (vlnka, tilda). Pro automatické do\-plnění vlnek slouží­ volně šiřitelný program \emph{vlna} od pana Olšáka\footnote{Viz http://petr.olsak.net/ftp/olsak/vlna/.}.

\end{document}